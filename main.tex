\documentclass{beamer}
\hypersetup{pdfpagemode=FullScreen}
\usepackage{beamerthemeshadow}
\usepackage{graphicx}
\usepackage{color}
\usepackage[utf8]{inputenc}
\usepackage{hyperref}
\usepackage[flushleft]{threeparttable}
\definecolor{beamer@red}{rgb}{0.4,0.6,0.9}
\setbeamercolor{structure}{fg=beamer@red}
\setbeamertemplate{navigation symbols}{} 
\usetheme{Berkeley}

\def\d{{\fontencoding{T1}\selectfont\dj}}
\def\D{{\fontencoding{T1}\selectfont\DJ}}

\begin{document}

\title{Džonatan Bouen}
\author{\flushleft Vladimir Jovanović \and\\ Andrea Mefailovski Stanojević \and\\ Marija Papović \and\\ Katarina Grbović }
\institute{\flushright Matematički fakultet\\Univerzitet u Beogradu}
\date{\footnotesize{Beograd, 2019.}}

\begin{frame}
	\thispagestyle{empty}
	\titlepage
\end{frame}

\addtocounter{framenumber}{-1}

\begin{frame}{Džonatan Piter Bouen}
    \begin{center} \includegraphics[scale=3]{Jonathan_Bowen_photograph.jpg} \end{center}
    \begin{itemize}
        \item{ Britanski stručnjak za računare }
        \item{ Predstavnik i osnivač "Museophile Limited" kompanije }
        \item{ Profesor emeritus na \textit{London South Bank} univerzitetu }
    \end{itemize}
\end{frame}

\begin{frame}
	\frametitle{Obrazovanje}
	\begin{itemize}
		\item Ro{\d}en je u Okfsordu 1956.
		\item Školovao se u "Dragon School" (Oksford) i u školi u Brajanstonu 
		\item Maturirao je na Univerzitetskom koledžu u Oksfordu, gde je stekao zvanje magistra inženjerskih nauka
		\item[]
		\item[] \begin{center} \includegraphics[scale=0.25]{University_College_Oxford_Coat_Of_Arms.png} \hspace{25} \includegraphics[scale=0.3]{Dragon_Coat_Of_Arms.png} \end{center}
	\end{itemize}
\end{frame}

\begin{frame}
\frametitle{Karijera}
    \begin{itemize}
        \item Radio na koledžu u Londonu
        \item Radio u kompjuterskoj laboratoriji univerziteta u Oksfordu
        \item Profesor na \textit{London South Bank} univerzitetu
    \end{itemize}
\end{frame}

\begin{frame}{Karijera}
    \begin{itemize}
        \item Svoj rad zasnivao na formalnim metodama i z-notacijama
        \item Predstavnik britanskog kompjuterskog društva
        \item Pomoćnik urednika za nekoliko novina o oblasti softverskog inženjerstva
	    \item[] \begin{left} \includegraphics[scale=2]{BCS-FACS_logo.jpg} \end{left}
    \end{itemize}
\end{frame}

\begin{frame}{Karijera}
    \begin{itemize}
		\item Osnovao je stranice virtualnih muzeja 1994. godine
		\item Virtualni muzej računarstva 
		\item Osnovao "Museophile Limited" 2002. godine
		\item  Radio je u kompanijama "Oxford Instruments", "Marconi Instruments", "Logica", "Silicon Graphics" i "Altran Praxis"
	\end{itemize}
\end{frame}

\begin{frame}{Karijera}
    \begin{itemize}
    		\item Izabran za člana RSA (\textit{Royal Society of Arts}) 2002. godine
    		\item Postao je član Britanskog kompjuterskog društva 2004. godine
    		\item Član organizacije "Worshipful Company of Information Technologists" u Londonu
    		\item[]
    		\item[] \begin{center} \includegraphics[scale=0.25]{7Nb1Hw2A_400x400.jpg} \end{center}
    \end{itemize}
\end{frame}

\begin{frame}[fragile]\frametitle{Literatura}
	\begin{itemize}
		\item Preuzeto sa: \href {https://sr.wikipedia.org/wiki/%D0%8F%D0%BE%D0%BD%D0%B0%D1%82%D0%B0%D0%BD_%D0%91%D0%BE%D1%83%D0%B5%D0%BD} {Wikipedia -- Džonatan Bouen}
	\end{itemize}
\end{frame}

\begin{frame}
    \begin{itemize}
        \center{\item[] \huge{HVALA NA PAŽNJI!}}
    \end{itemize}
\end{frame}

\end{document}
